\subsection{Тестирование}
\begin{center}
	\begin{longtable}{|p{0.3\linewidth}|p{0.3\linewidth}|p{0.3\linewidth}|}
		\hline
		\textbf{Вариант тестирования} & \textbf{Ожидаемый результат}&
		\textbf{Фактический результат}\\
		\hline
		\multicolumn{3}{|p{0.3\linewidth}|}{\textit{Пользователь}} \\
		\hline
		Пользователь вводит свои логин, пароль и нажимает кнопку <<Login>> & Пользователь авторизуется, на экране отображается содержимое документа и кнопки управления & На экране отобразилось содержимое документа и кнопки управления  \\
		\hline
		Пользователь вводит несуществующую пару логин, пароль & Будет показано сообщение об ошибке & Отображается сообщение об ошибке \\
		\hline
	Авторизоваться, если документ не заблокирован, то нажать кнопку <<Lock>> & Документ блокируется текущим пользователем, поле <<Lock owner>> & 
		В поле <<Lock owner>> отображается текущий пользователь \\
		\hline
		При заблокированном документе нажать кнопку <<Unlock>>
		& В поле <<Lock owner>> вместо текущего пользователя должно отобразиться значение null & В поле <<Lock owner>> вместо текущего пользователя отображается значение null \\
		\hline
		При заблокированном документе, внести изменение в текстовое поле, нажать кнопку <<Save>> и обновить страницу
		& После обновления страницы внесённые изменения должны сохраниться & После обновления страницы изменения документа сохранились \\
		\hline
		При незаблокированном документе нажать кнопку <<Save>>
		& Должно появиться сообщение о необходимости блокировки документа & Появляется сообщение о необходимости блокировки документа \\ \hline
		
	\end{longtable}
\end{center}