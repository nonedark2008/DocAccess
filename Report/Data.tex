\subsection{Слой хранения данных}
В качестве СУБД для хранения данных использовалась JavaDB. В соответствии с моделью предметной области, были выделены основные сущности: <<пользователь>> и <<документ>>. Для каждой из них были созданы соответствующие таблицы в базе данных.

\subsection{Слой бизнес-логики}
Средствами IDE NetBeans для сущностей <<Пользователь>> и <<Документ>> на основании таблиц в базе данных были автоматически сгенерированы соответствующие классы.

Для каждой из сущностей <<Пользователь>> и <<Документ>> были автоматически сгенерированы классы, соответствующие таблицам, созданным ранее в СУБД. Бизнес-логика расположена в пакете [\textit{docaccess}], содержащий следующие классы:
\begin{itemize}
	\item \textit{Users.java} - класс, соответствующий сущности <<Пользователь>>
	\item \textit{Documents.java} - класс, соответствующий сущности <<Документ>>
	\item \textit{DocAccessInterfaceRemote.java} - интерфейс взаимодействия удаленного пользователя с БД
	\item \textit{DocAccessInterface.java} - реализация интерфейса взаимодействия удаленного пользователя с БД
	
\end{itemize}